\documentclass[a4paper,12pt]{article}
\usepackage[a4paper]{geometry}
\usepackage{polski}
\usepackage[utf8]{inputenc}
\usepackage{graphicx}
\usepackage{enumitem}
\usepackage{amsmath}
\usepackage{amsfonts}
\usepackage{amsthm}
\usepackage{multicol}
\usepackage{multirow}
\usepackage{float}
\usepackage{url}
\usepackage{algorithm}
\usepackage{algpseudocode}
\usepackage[polish]{babel}
\usepackage{listings}

\theoremstyle{definition}
\newtheorem{definition}{Definicja}

\linespread{1.1}
\title{Wstęp do algorytmów ewolucyjnych \\ Projekt nr 8: \\ \textbf{Zmodyfikowany algorytm ewolucji różnicowej}\\
Raport z testów}

\begin{document}
	\maketitle
	\vfill
	\begin{flushright}
		\textbf{Autorzy}:\\
		Sylwia Nowak\\Katarzyna Węgiełek\\
	\end{flushright}
	
	\newpage
	\tableofcontents
	\newpage
	
\newgeometry{tmargin=3cm, bmargin=3cm, lmargin=3cm, rmargin=3cm}

\section{Treść zadania}
	Celem projektu była implementacja i testowanie algorytmu ewolucji różnicowej, w którym jako pierwszy z 3 punktów stosowanych podczas mutacji, wybierana jest średnia punktów populacji. Metoda została porównana z klasyczną wersją ewolucji różnicowej. Testy zostały przeprowadzone na benchmarku CEC 2013.
	
\section{Porównanie różnych wariantów selekcji pierwszego rodzica i krzyżowania}
Przeprowadzone testy miały za zadanie porównać skuteczność klasycznego algorytmu ewolucji różnicowej i jego zmodyfikowanej wersji. Zostały uwzględnione dwa typy selekcji pierwszego rodzica - wybór losowego osobnika z populacji (rand) oraz wybór najlepszego osobnika (best). Wykorzystane zostały też dwie metody krzyżowania - dwumianowe (bin) i wykładnicze (exp). 

Zatem obliczenia zostały wykonane dla następujących wariantów algorytmu:
\begin{itemize}
	\item DE/rand/1/bin
    \item DE/rand/1/exp
	\item DE/best/1/bin
    \item DE/best/1/exp
	\item DE/avg/1/bin
    \item DE/avg/1/exp
\end{itemize}
Ze względu na długi czas obliczeń została wprowadzona zmiana w stosunku do dokumentacji wstępnej -  zrezygnowałyśmy z przeprowadzania testów dla wariantów algorytmów, wykorzystujących 2 różnice wektorów w mutacji.

Dla każdego algorytmu zostały przeprowadzone testy z użyciem 28 funkcji z benchmarku CEC 2013. Dla każdej funkcji obliczenia zostały powtórzone 5 razy z tymi samymi wartościami parametrów. Tabele poniżej przedstawiają uśrednione wyniki z 5 uruchomień. Dla każdej wersji algorytmu podana została średnia wyników uzyskanych przez osobników z populacji po zakończeniu obliczeń oraz odpowiadający jej błąd, obliczany jako różnica między uzyskanym średnim wynikiem a optymalną wartością dla danej funkcji benchmarku.\\

Wszystkie obliczenia zostały przeprowadzone dla populacji złożonej z 50 osobników, zainicjowanej według jednostajnego rozkładu losowego w przestrzeni poszukiwań (przedział $[-100,100]^D$). Rozpatrywane zostały problemy o wymiarowości 2. Algorytm kończył pracę po wykonaniu 1000 iteracji. Przyjęte zostały wartości parametrów $F = 0.5$ i $c_r = 0.5$.

\begin{table}[H]
\centering
\def\arraystretch{1.3}
\setlength\tabcolsep{10pt}
\caption{Porównanie algorytmów \textbf{DE/rand/1/bin} i \textbf{DE/avg/1/bin}}
\vspace{8pt}
\begin{tabular}{|c|c|c|c|c|c|c|}
	
	\hline
	\multirow{2}{*}{Nr funkcji} & \multicolumn{2}{c|}{DE/rand/1/bin} & \multicolumn{2}{c|}{DE/avg/1/bin}\\
	\cline{2-5}
	& średni wynik & błąd & średni wynik & błąd\\\hline
	1     & -1400 & 0     & -1400 & 0 \\\hline
    2     & -1300 & 0     & -1300 & 0 \\\hline
    3     & -1200 & 0     & -1024,86 & 175,14 \\\hline
    4     & -1100 & 0     & -888,0326 & 211,9674 \\\hline
    5     & -1000 & 0     & -999,9958 & 0,0042 \\\hline
    6     & -900  & 0     & -900  & 0 \\\hline
    7     & -711,6652 & 88,3348 & -800  & 0 \\\hline
    8     & -679,9703 & 20,0297 & -699,9742 & 0,0258 \\\hline
    9     & -600  & 0     & -599,4461 & 0,5539 \\\hline
    10    & -500  & 0     & -499,9015 & 0,0985 \\\hline
    11    & -400  & 0     & -396,2258 & 3,7742 \\\hline
    12    & -299,005 & 0,995 & -299,9985 & 0,0015 \\\hline
    13    & -198,0101 & 1,9899 & -200  & 0 \\\hline
    14    & -99,68783 & 0,31217 & -96,71519 & 3,28481 \\\hline
    15    & 158,5045 & 58,5045 & 100,3123 & 0,3123 \\\hline
    16    & 200   & 0     & 205,3461 & 5,3461 \\\hline
    17    & 302,0196 & 2,0196 & 305,4988 & 5,4988 \\\hline
    18    & 400   & 0     & 405,4569 & 5,4569 \\\hline
    19    & 500   & 0     & 500,1756 & 0,1756 \\\hline
    20    & 600,0194 & 0,0194 & 600,0195 & 0,0195 \\\hline
    21    & 700   & 0     & 910,2265 & 210,2265 \\\hline
    22    & 1473,798 & 673,798 & 841,1363 & 41,1363 \\\hline
    23    & 900   & 0     & 900,0001 & 1E-04 \\\hline
    24    & 1000  & 0     & 1000  & 0 \\\hline
    25    & 1200  & 100   & 1184,501 & 84,501 \\\hline
    26    & 1200  & 0     & 1201,754 & 1,754 \\\hline
    27    & 2053,969 & 753,969 & 1644,641 & 344,641 \\\hline
    28    & 2518,616 & 1118,616 & 1400,311 & 0,311 \\\hline
\end{tabular}	
\end{table}

\begin{table}[H]
\centering
\def\arraystretch{1.3}
\setlength\tabcolsep{10pt}
\caption{Porównanie algorytmów \textbf{DE/best/1/bin} i \textbf{DE/avg/1/bin}}
\vspace{8pt}
\begin{tabular}{|c|c|c|c|c|c|c|}
	
	\hline
	\multirow{2}{*}{Nr funkcji} & \multicolumn{2}{c|}{DE/best/1/bin} & \multicolumn{2}{c|}{DE/avg/1/bin}\\
	\cline{2-5}
	& średni wynik & błąd & średni wynik & błąd\\\hline
	1     & -1400 & 0     & -1400 & 0 \\\hline
    2     & -1300 & 0     & -1300 & 0 \\\hline
    3     & -1200 & 0     & -1024,86 & 175,14 \\\hline
    4     & -1100 & 0     & -888,0326 & 211,9674 \\\hline
    5     & -1000 & 0     & -999,9958 & 0,0042 \\\hline
    6     & -900  & 0     & -900  & 0 \\\hline
    7     & -800  & 0     & -800  & 0 \\\hline
    8     & -700  & 0     & -699,9742 & 0,0258 \\\hline
    9     & -599,9927 & 0,0073 & -599,4461 & 0,5539 \\\hline
    10    & -500  & 0     & -499,9015 & 0,0985 \\\hline
    11    & -400  & 0     & -396,2258 & 3,7742 \\\hline
    12    & -300  & 0     & -299,9985 & 0,0015 \\\hline
    13    & -200  & 0     & -200  & 0 \\\hline
    14    & -99,37565 & 0,62435 & -96,71519 & 3,28481 \\\hline
    15    & 117,0694 & 17,0694 & 100,3123 & 0,3123 \\\hline
    16    & 200   & 0     & 205,3461 & 5,3461 \\\hline
    17    & 303,7462 & 3,7462 & 305,4988 & 5,4988 \\\hline
    18    & 402,0035 & 2,0035 & 405,4569 & 5,4569 \\\hline
    19    & 500   & 0     & 500,1756 & 0,1756 \\\hline
    20    & 600   & 0     & 600,0195 & 0,0195 \\\hline
    21    & 900   & 200   & 910,2265 & 210,2265 \\\hline
    22    & 923,9188 & 123,9188 & 841,1363 & 41,1363 \\\hline
    23    & 978,7674 & 78,7674 & 900,0001 & 1E-04 \\\hline
    24    & 1000  & 0     & 1000  & 0 \\\hline
    25    & 1102,194 & 2,194 & 1184,501 & 84,501 \\\hline
    26    & 1200  & 0     & 1201,754 & 1,754 \\\hline
    27    & 1400  & 100   & 1644,641 & 344,641 \\\hline
    28    & 1500  & 100   & 1400,311 & 0,311 \\\hline
\end{tabular}	
\end{table}

\begin{table}[H]
\centering
\def\arraystretch{1.3}
\setlength\tabcolsep{10pt}
\caption{Porównanie algorytmów \textbf{DE/rand/1/exp} i \textbf{DE/avg/1/exp}}
\vspace{8pt}
\begin{tabular}{|c|c|c|c|c|c|c|}
	
	\hline
	\multirow{2}{*}{Nr funkcji} & \multicolumn{2}{c|}{DE/rand/1/exp} & \multicolumn{2}{c|}{DE/avg/1/exp}\\
	\cline{2-5}
	& średni wynik & błąd & średni wynik & błąd\\\hline
	1     & -1400 & 0     & -1399,997 & 0,003 \\\hline
    2     & -1300 & 0     & -1299,939 & 0,061 \\\hline
    3     & -1200 & 0     & -959,4715 & 240,5285 \\\hline
    4     & -1100 & 0     & -816,6373 & 283,3627 \\\hline
    5     & -1000 & 0     & -999,9998 & 0,0002 \\\hline
    6     & -900  & 0     & -899,9999 & 1E-04 \\\hline
    7     & -711,6652 & 88,3348 & -799,4801 & 0,5199 \\\hline
    8     & -679,9997 & 20,0003 & -699,9954 & 0,0046 \\\hline
    9     & -597  & 3     & -599,9893 & 0,0107 \\\hline
    10    & -500  & 0     & -499,7222 & 0,2778 \\\hline
    11    & -400  & 0     & -400  & 0 \\\hline
    12    & -295,0252 & 4,9748 & -300  & 0 \\\hline
    13    & -200  & 0     & -197,2395 & 2,7605 \\\hline
    14    & 458,5593 & 558,5593 & -79,96757 & 20,03243 \\\hline
    15    & 174,9496 & 74,9496 & 122,652 & 22,652 \\\hline
    16    & 200   & 0     & 207,7793 & 7,7793 \\\hline
    17    & 300   & 0     & 304,0274 & 4,0274 \\\hline
    18    & 400   & 0     & 405,0404 & 5,0404 \\\hline
    19    & 500   & 0     & 500,1277 & 0,1277 \\\hline
    20    & 600,0194 & 0,0194 & 600,0195 & 0,0195 \\\hline
    21    & 900   & 200   & 700,0263 & 0,0263 \\\hline
    22    & 1473,798 & 673,798 & 851,2916 & 51,2916 \\\hline
    23    & 1440,659 & 540,659 & 916,1321 & 16,1321 \\\hline
    24    & 1129,194 & 129,194 & 1000  & 0 \\\hline
    25    & 1300  & 200   & 1100  & 0 \\\hline
    26    & 1200,081 & 0,081 & 1205,136 & 5,136 \\\hline
    27    & 2349,932 & 1049,932 & 1613,554 & 313,554 \\\hline
    28    & 2518,616 & 1118,616 & 1400,005 & 0,005 \\\hline
\end{tabular}	
\end{table}

\begin{table}[H]
\centering
\def\arraystretch{1.3}
\setlength\tabcolsep{10pt}
\caption{Porównanie algorytmów \textbf{DE/best/1/exp} i \textbf{DE/avg/1/exp}}
\vspace{8pt}
\begin{tabular}{|c|c|c|c|c|c|c|}
	
	\hline
	\multirow{2}{*}{Nr funkcji} & \multicolumn{2}{c|}{DE/best/1/exp} & \multicolumn{2}{c|}{DE/avg/1/exp}\\
	\cline{2-5}
	& średni wynik & błąd & średni wynik & błąd\\\hline
	1     & -1400 & 0     & -1399,997 & 0,003 \\\hline
    2     & -1300 & 0     & -1299,939 & 0,061 \\\hline
    3     & -1200 & 0     & -959,4715 & 240,5285 \\\hline
    4     & -1100 & 0     & -816,6373 & 283,3627 \\\hline
    5     & -1000 & 0     & -999,9998 & 0,0002 \\\hline
    6     & -900  & 0     & -899,9999 & 1E-04 \\\hline
    7     & -800  & 0     & -799,4801 & 0,5199 \\\hline
    8     & -700  & 0     & -699,9954 & 0,0046 \\\hline
    9     & -600  & 0     & -599,9893 & 0,0107 \\\hline
    10    & -499,9926 & 0,0074 & -499,7222 & 0,2778 \\\hline
    11    & -400  & 0     & -400  & 0 \\\hline
    12    & -300  & 0     & -300  & 0 \\\hline
    13    & -200  & 0     & -197,2395 & 2,7605 \\\hline
    14    & -66,48556 & 33,51444 & -79,96757 & 20,03243 \\\hline
    15    & 100,3122 & 0,3122 & 122,652 & 22,652 \\\hline
    16    & 200   & 0     & 207,7793 & 7,7793 \\\hline
    17    & 302,0196 & 2,0196 & 304,0274 & 4,0274 \\\hline
    18    & 400   & 0     & 405,0404 & 5,0404 \\\hline
    19    & 500,0632 & 0,0632 & 500,1277 & 0,1277 \\\hline
    20    & 600,0419 & 0,0419 & 600,0195 & 0,0195 \\\hline
    21    & 711,3752 & 11,3752 & 700,0263 & 0,0263 \\\hline
    22    & 917,3027 & 117,3027 & 851,2916 & 51,2916 \\\hline
    23    & 903,852 & 3,852 & 916,1321 & 16,1321 \\\hline
    24    & 1000  & 0     & 1000  & 0 \\\hline
    25    & 1101,7621 & 1,7621 & 1100  & 0 \\\hline
    26    & 1200  & 0     & 1205,136 & 5,136 \\\hline
    27    & 1387,3285 & 87,3285 & 1613,554 & 313,554 \\\hline
    28    & 1510,4932 & 110,4932 & 1400,005 & 0,005 \\\hline
	\hline
\end{tabular}	
\end{table}

\section{Porównanie różnych wartości parametru F}
Przetestowany został również wpływ parametru F na skuteczność algorytmu ewolucji różnicowej. Przeprowadzone testy uwzględniały trzy różne wartości tego parametru: $0.2$, $0.5$ oraz $0.8$. Obiczenia zostały przeprowadzone dla klasycznej wersji algorytmu ewolucji różnicowej, wykorzystującej selekcję losową i krzyżowanie dwumianowe (\textbf{DE/rand/1/bin}) oraz dla wersji zmodyfikowanej, również korzystającej z krzyżowania dwumianowego (\textbf{DE/avg/1/bin}). Poniższe tabele przedstawiają średnią wyników uzyskanych przez punkty populacji po zakończeniu obliczeń oraz błąd, obliczany jako różnica między średnim wynikiem a wartością optymalną dla danej funkcji benchmarku.\\

Podobnie jak poprzednio testy zostały przeprowadzone dla populacji złożonej z 50 osobników, wymiarowość problemu wynosiła 2, a parametr $c_r = 0.5$ Obliczenia zatrzymywały się po wykonaniu 1000 iteracji. Dla każdej z 28 funkcji testy zostały powtórzone 5 razy, wyniki przedstawione w tabelach są średnią z 5 uruchomień dla danej funkcji.

\begin{table}[H]
\centering
\def\arraystretch{1.3}
\setlength\tabcolsep{10pt}
\caption{Porównanie wyników dla różnych wartości parametru F dla algorytmu \textbf{klasycznego}}
\vspace{8pt}
\begin{tabular}{|c|c|c|c|c|c|c|c|c|c|}
	\hline
	\multirow{2}{*}{Nr funkcji} & \multicolumn{2}{c|}{F = 0.2} & \multicolumn{2}{c|}{F = 0.5} & \multicolumn{2}{c|}{F = 0.8}\\
	\cline{2-7}
	& średnia & błąd & średnia & błąd & średnia & błąd \\\hline
	1     & -1400 & 0     & -1400 & 0     & -1400 & 0 \\\hline
    2     & -1300 & 0     & -1300 & 0     & -1300 & 0 \\\hline
    3     & -1200 & 0     & -1200 & 0     & -1200 & 0 \\\hline
    4     & -1100 & 0     & -1100 & 0     & -1100 & 0 \\\hline
    5     & -1000 & 0     & -1000 & 0     & -1000 & 0 \\\hline
    6     & -899,9946 & 0,0054 & -900  & 0     & -900  & 0 \\\hline
    7     & -800  & 0     & -800  & 0     & -800  & 0 \\\hline
    8     & -700  & 0     & -700  & 0     & -700  & 0 \\\hline
    9     & -600  & 0     & -600  & 0     & -600  & 0 \\\hline
    10    & -499,9901 & 0,0099 & -500  & 0     & -499,9926 & 0,0074 \\\hline
    11    & -400  & 0     & -400  & 0     & -400  & 0 \\\hline
    12    & -300  & 0     & -300  & 0     & -300  & 0 \\\hline
    13    & -200  & 0     & -200  & 0     & -200  & 0 \\\hline
    14    & -100  & 0     & -100  & 0     & -99,37565 & 0,62435 \\\hline
    15    & 100,3122 & 0,3122 & 100,3122 & 0,3122 & 100   & 0 \\\hline
    16    & 200,0007 & 0,0007 & 200   & 0     & 200,067 & 0,067 \\\hline
    17    & 300,0626 & 0,0626 & 302,0196 & 2,0196 & 302,0196 & 2,0196 \\\hline
    18    & 402,033 & 2,033 & 400   & 0     & 400,2014 & 0,2014 \\\hline
    19    & 500   & 0     & 500   & 0     & 500   & 0 \\\hline
    20    & 600,0194 & 0,0194 & 600,0194 & 0,0194 & 600   & 0 \\\hline
    21    & 700   & 0     & 700   & 0     & 700   & 0 \\\hline
    22    & 801,6223 & 1,6223 & 800   & 0     & 800   & 0 \\\hline
    23    & 900   & 0     & 900   & 0     & 900   & 0 \\\hline
    24    & 1004,843 & 4,843 & 1000  & 0     & 1000  & 0 \\\hline
    25    & 1100  & 0     & 1100  & 0     & 1100  & 0 \\\hline
    26    & 1200  & 0     & 1200  & 0     & 1200  & 0 \\\hline
    27    & 1410,723 & 110,723 & 1400  & 100   & 1300  & 0 \\\hline
    28    & 1406,662 & 6,662 & 1400  & 0     & 1400  & 0 \\\hline
\end{tabular}	
\end{table}

\begin{table}[H]
\centering
\def\arraystretch{1.3}
\setlength\tabcolsep{10pt}
\caption{Porównanie wyników dla różnych wartości parametru F dla algorytmu \textbf{zmodyfikowanego}}
\vspace{8pt}
\begin{tabular}{|c|c|c|c|c|c|c|c|c|c|}
	
	\hline
	\multirow{2}{*}{Nr funkcji} & \multicolumn{2}{c|}{F = 0.2} & \multicolumn{2}{c|}{F = 0.5} & \multicolumn{2}{c|}{F = 0.8}\\
	\cline{2-7}
	& średnia & błąd & średnia & błąd & średnia & błąd \\\hline
	1     & -1397,534 & 2,466 & -1399,995 & 0,005 & -1400 & 0 \\\hline
    2     & 23245,75 & 24545,75 & -1295,273 & 4,727 & -1300 & 0 \\\hline
    3     & 1116116 & 1117316 & -978,0623 & 221,9377 & -1200 & 0 \\\hline
    4     & 181164,5 & 182264,5 & -1100 & 0     & -1100 & 0 \\\hline
    5     & -996,7731 & 3,2269 & -999,9972 & 0,0028 & -1000 & 0 \\\hline
    6     & -899,9988 & 0,0012 & -900  & 0     & -900  & 0 \\\hline
    7     & -797,4983 & 2,5017 & -799,9829 & 0,0171 & -800  & 0 \\\hline
    8     & -691,5221 & 8,4779 & -700  & 0     & -700  & 0 \\\hline
    9     & -599,0594 & 0,9406 & -600  & 0     & -598,9099 & 1,0901 \\\hline
    10    & -499,0636 & 0,9364 & -499,8942 & 0,1058 & -499,9522 & 0,0478 \\\hline
    11    & -399,7083 & 0,2917 & -400  & 0     & -395,7877 & 4,2123 \\\hline
    12    & -297,9883 & 2,0117 & -296,0832 & 3,9168 & -296,6161 & 3,3839 \\\hline
    13    & -196,4516 & 3,5484 & -196,4791 & 3,5209 & -200  & 0 \\\hline
    14    & -99,21282 & 0,78718 & -65,42365 & 34,57635 & -45,01767 & 54,98233 \\\hline
    15    & 117,8144 & 17,8144 & 130,4853 & 30,4853 & 100,6243 & 0,6243 \\\hline
    16    & 204,3826 & 4,3826 & 206,4275 & 6,4275 & 203,0594 & 3,0594 \\\hline
    17    & 308,5668 & 8,5668 & 304,774 & 4,774 & 302,8151 & 2,8151 \\\hline
    18    & 407,6667 & 7,6667 & 404,6784 & 4,6784 & 402,826 & 2,826 \\\hline
    19    & 500,1421 & 0,1421 & 500,1223 & 0,1223 & 500,0152 & 0,0152 \\\hline
    20    & 600,0194 & 0,0194 & 600,0194 & 0,0194 & 600,0195 & 0,0195 \\\hline
    21    & 918,9835 & 218,9835 & 910,7025 & 210,7025 & 700   & 0 \\\hline
    22    & 865,5707 & 65,5707 & 851,7775 & 51,7775 & 922,7553 & 122,7553 \\\hline
    23    & 1150,765 & 250,765 & 900,1163 & 0,1163 & 961,9144 & 61,9144 \\\hline
    24    & 1074,374 & 74,374 & 1000  & 0     & 1009,897 & 9,897 \\\hline
    25    & 1173,856 & 73,856 & 1100,936 & 0,936 & 1187,672 & 87,672 \\\hline
    26    & 1261,403 & 61,403 & 1200  & 0     & 1216,987 & 16,987 \\\hline
    27    & 1655,366 & 355,366 & 1630,332 & 330,332 & 1445,497 & 145,497 \\\hline
    28    & 1458,961 & 58,961 & 1507,966 & 107,966 & 1506,462 & 106,462 \\\hline
\end{tabular}	
\end{table}

\section{Porównanie różnych wartości parametru Cr}
Zostało zbadane także działanie algorytmu ewolucji różnicowej dla różnych wartości parametru $c_r$. Przeprowadzone testy uwzględniały trzy różne wartości tego parametru: $0.2$, $0.5$ oraz $0.8$. Tak jak dla paprametru F, obliczenia zostały przeprowadzone z wykorzystaniem dwóch wersji algorytmu - klasycznej, wykorzystującej selekcję losową i krzyżowanie dwumianowe (\textbf{DE/rand/1/bin}) oraz zmodyfikowanej, korzystającej z krzyżowania dwumianowego (\textbf{DE/avg/1/bin}). Poniższe tabele przedstawiają średnią wyników uzyskanych przez punkty populacji po zakończeniu obliczeń oraz błąd, obliczany jako różnica między średnim wynikiem a wartością optymalną dla danej funkcji benchmarku.\\

Podobnie jak poprzednio testy zostały przeprowadzone dla populacji złożonej z 50 osobników, wymiarowość problemu wynosiła 2. Parametr F był równy 0.5. Obliczenia zatrzymywały się po wykonaniu 1000 iteracji. Dla każdej z 28 funkcji testy zostały powtórzone 5 razy, wyniki przedstawione w tabelach są średnią z 5 uruchomień dla danej funkcji.

\begin{table}[H]
\centering
\def\arraystretch{1.3}
\setlength\tabcolsep{10pt}
\caption{Porównanie wyników dla różnych wartości parametru $C_r$ dla algorytmu \textbf{klasycznego}}
\vspace{8pt}
\begin{tabular}{|c|c|c|c|c|c|c|c|c|c|}
	
	\hline
	\multirow{2}{*}{Nr funkcji} & \multicolumn{2}{c|}{$C_r$ = 0.2} & \multicolumn{2}{c|}{$C_r$ = 0.5} & \multicolumn{2}{c|}{$C_r$ = 0.8}\\
	\cline{2-7}
	& średnia & błąd & średnia & błąd & średnia & błąd \\\hline
	1     & -1400 & 0     & -1400 & 0     & -1400 & 0 \\\hline
    2     & -1300 & 0     & -1300 & 0     & -1300 & 0 \\\hline
    3     & -1200 & 0     & -1200 & 0     & -1200 & 0 \\\hline
    4     & -1100 & 0     & -1100 & 0     & -1100 & 0 \\\hline
    5     & -1000 & 0     & -1000 & 0     & -1000 & 0 \\\hline
    6     & -900  & 0     & -900  & 0     & -900  & 0 \\\hline
    7     & -800  & 0     & -800  & 0     & -800  & 0 \\\hline
    8     & -700  & 0     & -700  & 0     & -700  & 0 \\\hline
    9     & -600  & 0     & -600  & 0     & -600  & 0 \\\hline
    10    & -500  & 0     & -500  & 0     & -500  & 0 \\\hline
    11    & -400  & 0     & -400  & 0     & -400  & 0 \\\hline
    12    & -300  & 0     & -300  & 0     & -300  & 0 \\\hline
    13    & -200  & 0     & -200  & 0     & -200  & 0 \\\hline
    14    & -99,68783 & 0,31217 & -99,68783 & 0,31217 & -100  & 0 \\\hline
    15    & 100,3122 & 0,3122 & 100,3122 & 0,3122 & 100   & 0 \\\hline
    16    & 200,0001 & 0,0001 & 200   & 0     & 200   & 0 \\\hline
    17    & 302,0196 & 2,0196 & 300   & 0     & 302,0196 & 2,0196 \\\hline
    18    & 402,0021 & 2,0021 & 400   & 0     & 402,0021 & 2,0021 \\\hline
    19    & 500   & 0     & 500   & 0     & 500   & 0 \\\hline
    20    & 600   & 0     & 600   & 0     & 600   & 0 \\\hline
    21    & 700   & 0     & 700   & 0     & 700   & 0 \\\hline
    22    & 800   & 0     & 800   & 0     & 800   & 0 \\\hline
    23    & 900   & 0     & 900   & 0     & 900   & 0 \\\hline
    24    & 1000  & 0     & 1000  & 0     & 1003,147 & 3,147 \\\hline
    25    & 1100  & 0     & 1200  & 100   & 1100  & 0 \\\hline
    26    & 1200,081 & 0,081 & 1200  & 0     & 1200  & 0 \\\hline
    27    & 1400  & 100   & 1300  & 0     & 1400  & 100 \\\hline
    28    & 1400  & 0     & 1400  & 0     & 1400  & 0 \\\hline
\end{tabular}	
\end{table}

\begin{table}[H]
\centering
\def\arraystretch{1.3}
\setlength\tabcolsep{10pt}
\caption{Porównanie wyników dla różnych wartości parametru $C_r$ dla algorytmu \textbf{zmodyfikowanego}}
\vspace{8pt}
\begin{tabular}{|c|c|c|c|c|c|c|c|c|c|}
	
	\hline
	\multirow{2}{*}{Nr funkcji} & \multicolumn{2}{c|}{$C_r$ = 0.2} & \multicolumn{2}{c|}{$C_r$ = 0.5} & \multicolumn{2}{c|}{$C_r$ = 0.8}\\
	\cline{2-7}
	& średnia & błąd & średnia & błąd & średnia & błąd \\\hline
	1     & -1400 & 0     & -1400 & 0     & -1400 & 0 \\\hline
    2     & -1295,638 & 4,362 & -1298,699 & 1,301 & -1253,495 & 46,505 \\\hline
    3     & -834,6177 & 365,3823 & -507,5096 & 692,4904 & -1005,128 & 194,872 \\\hline
    4     & -1097,988 & 2,012 & -1067,658 & 32,342 & -1096,269 & 3,731 \\\hline
    5     & -999,9991 & 0,0009 & -1000 & 0     & -999,99 & 0,01 \\\hline
    6     & -900  & 0     & -899,9999 & 1E-04 & -900  & 0 \\\hline
    7     & -799,6433 & 0,3567 & -790,4417 & 9,5583 & -798,1647 & 1,8353 \\\hline
    8     & -697,424 & 2,576 & -699,7256 & 0,2744 & -699,9502 & 0,0498 \\\hline
    9     & -599,9921 & 0,0079 & -599,8844 & 0,1156 & -599,9269 & 0,0731 \\\hline
    10    & -499,8747 & 0,1253 & -499,8835 & 0,1165 & -499,8527 & 0,1473 \\\hline
    11    & -398,7644 & 1,2356 & -397,0982 & 2,9018 & -392,9514 & 7,0486 \\\hline
    12    & -300  & 0     & -298,8926 & 1,1074 & -298,9142 & 1,0858 \\\hline
    13    & -196,4788 & 3,5212 & -200  & 0     & -196,6559 & 3,3441 \\\hline
    14    & -59,89997 & 40,10003 & -70,75123 & 29,24877 & -63,93472 & 36,06528 \\\hline
    15    & 146,4051 & 46,4051 & 118,7363 & 18,7363 & 112,9297 & 12,9297 \\\hline
    16    & 207,1146 & 7,1146 & 207,0848 & 7,0848 & 207,4706 & 7,4706 \\\hline
    17    & 303,7807 & 3,7807 & 304,5201 & 4,5201 & 304,4148 & 4,4148 \\\hline
    18    & 404,5338 & 4,5338 & 404,8775 & 4,8775 & 404,437 & 4,437 \\\hline
    19    & 500   & 0     & 500,0629 & 0,0629 & 500,0197 & 0,0197 \\\hline
    20    & 600,0195 & 0,0195 & 600,0195 & 0,0195 & 600,0194 & 0,0194 \\\hline
    21    & 903,0204 & 203,0204 & 703,5236 & 3,5236 & 700,0023 & 0,0023 \\\hline
    22    & 841,8824 & 41,8824 & 856,9928 & 56,9928 & 800   & 0 \\\hline
    23    & 916,6188 & 16,6188 & 916,1321 & 16,1321 & 900,5693 & 0,5693 \\\hline
    24    & 1000  & 0     & 1000,009 & 0,009 & 1007,098 & 7,098 \\\hline
    25    & 1204,077 & 104,077 & 1199,718 & 99,718 & 1205,81 & 105,81 \\\hline
    26    & 1200,081 & 0,081 & 1200,381 & 0,381 & 1200,296 & 0,296 \\\hline
    27    & 1425,116 & 125,116 & 1618,015 & 318,015 & 1635,926 & 335,926 \\\hline
    28    & 1504,179 & 104,179 & 1504,139 & 104,139 & 1505,076 & 105,076 \\\hline
\end{tabular}	
\end{table}

\end{document}
